\documentclass[final]{article}


% if you need to pass options to natbib, use, e.g.:
%     \PassOptionsToPackage{numbers, compress}{natbib}
% before loading neurips_2024


% ready for submission
\usepackage{neurips_2024}


% to compile a preprint version, e.g., for submission to arXiv, add add the
% [preprint] option:
%     \usepackage[preprint]{neurips_2024}


% to compile a camera-ready version, add the [final] option, e.g.:
%     \usepackage[final]{neurips_2024}


% to avoid loading the natbib package, add option nonatbib:
%    \usepackage[nonatbib]{neurips_2024}


\usepackage[utf8]{inputenc} % allow utf-8 input
\usepackage[T1]{fontenc}    % use 8-bit T1 fonts
\usepackage{hyperref}       % hyperlinks
\usepackage{url}            % simple URL typesetting
\usepackage{booktabs}       % professional-quality tables
\usepackage{amsfonts}       % blackboard math symbols
\usepackage{nicefrac}       % compact symbols for 1/2, etc.
\usepackage{microtype}      % microtypography
\usepackage{xcolor}         % colors


\title{AI boom and AI effect}
\author{
	Jakkula Adishesh Balaji \\
	AI24BTECH11016 \\ 
	}

\begin{document}
\maketitle

\section{AI boom}
The AI boom has begun in the late 2010s and gained worldwide attention in March of 2016 when a computer program AlphaGo developed by Deepmind beat a 9-dan professional in the game Go. Technologies such as alphafold revolutionized the process of drug development due to more accurate predictions in protein-folding. Interestingly, AI could predict the shape of the proteins accurately to the width of an atom. By 2022, LLMs(Large Language Models) were capable of producing text to image models that could generate images using a text prompt. The generative AI race began in 2016 which followed the advances made in GPUs. The market capitalization of NVidia, a company whose GPUs are in high demand to train and use generative AI models, rose to over US\$ 3.3 Trillion. GPT-3 is an LLM released in 2020 by OpenAI which is capable of generating human-like text. Its upgraded version GPT-3.5 could produce articulate answers of very high quality and performed well on college entrance exams like SAT. Big-Tech viewed AI as a threat to various search engines. Across various industries, AI tools are being abundant and are proven to cause incremental change in their efficiency and caused positive influence on their revenue. There are concerns that AI could threaten job security in the future, due to the increased availability and affordability of AI powered machines. It can also increase the speed and stealth of cyberattacks. AI also has the ability to generate convincing messages which can facilitate large-scale propaganda or misinformation. Ultimately, the benefits of AI outweigh its risks.
\section{AI effect}
According to Stottler Henke, "The great practical benefits of AI applications go largely unnoticed by many despite the already widespread use of AI techniques in software. This is the AI effect phenomenon. When AI systems perform tasks which were previously considered as intelligent with great proficiency, the human perception of intelligence is shifted to exclude those certain tasks, this is an example of shifting the goalposts. This phenomenon led AI advancements to be dismissed as mere computations and not "real intelligence". AI applications have been integrated into many mainstream tasks, often not being given recognition. For example, when IBM's DeepBlue defeated the reigning World Champion in chess(Garry Kasparov), people criticized the invention saying it is just a brute-force machine which did not have a deep understanding of the game. Whenever a machine does something intelligent, it ceases to be regarded as intelligent. This phenomenon is linked to efforts to preserve human uniqueness and the changing perceptions of what constitutes intelligence. Some experts think that the AI effect might continue, with advances in AI endlessly changing the public perceptions and definitions of the term AI. Understanding the AI effect helps in contextualizing how we perceive and value intelligence, both artificial and human. 
\section{References}
\url{https://en.wikipedia.org/wiki/AI_boom} \\
\url{https://en.wikipedia.org/wiki/AI_effect}


\end{document}

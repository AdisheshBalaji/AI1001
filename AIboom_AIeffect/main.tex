\documentclass[journal,12pt,onecolumn]{IEEEtran}
\usepackage[utf8]{inputenc}
\usepackage[a5paper, margin=10mm, onecolumn]{geometry}
\setlength{\headheight}{1cm} 
\setlength{\headsep}{0mm}    

\title{AI boom and AI effect}
\author{Adishesh Balaji}
\date{September 2024}
\linespread{1.5}
\begin{document}
\bibliographystyle{IEEEtran}
\maketitle
\section{\textbf{AI boom}}
The AI boom has begun in the late 2010s and gained worldwide attention in March of 2016 when a computer program \textbf{AlphaGo} developed by Deepmind beat a 9-dan professional in the game Go. Technologies such as alphafold revolutionized the process of drug development due to more accurate predictions in protein-folding. Interestingly, AI could predict the shape of the proteins accurately to the width of an atom.
\textbf{GPT-3} is an LLM released by \textbf{OpenAI} in 2023 and is capable of generating human-like text which could theoretically pass the \textbf{turing test}. The market capitalization of \textbf{NVidia}, a company whose GPUs are in high demand to train and use generative AI models, rose to over US\$ 3.3 Trillion.
\section{\textbf{AI effect}}
According to Stottler Henke, "The great practical benefits of AI applications go largely unnoticed by many despite the already widespread use of AI techniques in software. This is the AI effect. This effect led AI advancements to be dismissed as mere computations and not "real intelligence". AI applications have been integrated into many widespread tasks, often not being given recognition. 



\end{document}
